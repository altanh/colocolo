% Thomas Bernardi
% University of Washington
% HLIN501 Study Guide
% November 2017

\documentclass{article}
\usepackage[T1]{fontenc}
% \usepackage{babel}
\usepackage[margin=1in]{geometry}
\usepackage{parskip}
\usepackage{amsthm}
\usepackage{amsmath}
\usepackage{tabularx}
\usepackage{graphicx}
\usepackage{enumitem}
\graphicspath{ {./graphs/} }
\usepackage[noend]{algorithmic}
\usepackage{amsmath, amsfonts, amsthm, amssymb} 
\usepackage{listings}

\theoremstyle{definition}
\newtheorem{definition}{Definition}[section]
\newtheorem{theorem}{Theorem}[section]
\title{Colocolo\\\large Optimizations on Ocelot \\ A Racket backend for Alloy}
\date{February 2019}
\author{Thomas Bernardi, Altan Haan}

\DeclareMathOperator{\bigdot}{\bullet}

\begin{document}
	\maketitle
	\begin{center}	
		\textit{Everything is a lie, nothing is real.} \\
    \end{center}

    \section{Ocelot}

    \section{Skolemization}

    To distinguish between the syntax of quantifiers and their semantics we introduce two new terms:\\
    \theoremstyle{definition}
    \begin{definition}{Weak quantifier}
        A \textbf{weak quantifier} is one that \emph{semantically} acts as an existential quantifier (i.e. $\exists$ and $\neg \forall$).
    \end{definition}

    \begin{definition}{Strong quantifier}
        A \textbf{weak quantifier} is one that \emph{semantically} acts as an existential quantifier (i.e. $\forall$ and $\neg \exists$).
    \end{definition}

    Skolemization is a transformation that moves weak quantifiers to the highest syntactic level by replacing weakly quantified variables with functions (or relations in the case of alloy). Skolemization relies on a key second order equivalence:

    \begin{equation} \label{eqv1}
        \forall x \exists y~.~P(x, y) \Leftrightarrow \exists f \forall x~.~P(x, f(x))
    \end{equation}

    where $f(x)$ is a function (relation) that maps $x$ to $y$. Intuitively this is translating between ``forall $x$ there exists a $y$ such that $P(x, y)$'' into ``there exists a function $f : x \rightarrow y$ such that, forall $x$, $P(x, f(x))$.'' This is useful because the existence of $f$ is semantically equivalent to the satisfiability of the formula if $f$ is a variable sent to the solver.

    In the case of bounded model checking (in both KodKod and Ocelot), this significantly reduces formula size. Suppose we have the universe $\{A_1, \dots, A_n\}$. Considering the formula from our equivalence \ref{eqv1}, $\forall x \exists y~.~P(x,y)$ would get translated as follows:

    Without skolemization:
    \begin{equation} \label{woskolem}
        \bigwedge_{i=1}^n\bigvee_{j=1}^n P(A_i, A_j)
    \end{equation}

    With skolemization:

    \begin{equation} \label{woskolem}
        \bigwedge_{i=1}^nP(A_i, f(A_i))
    \end{equation}

    \subsection{The Algorithm}

    \theoremstyle{definition}
    \begin{definition}{Declaration}
        A declaration allows us to define variables in terms of other expressions (and all declarations can eventually be traced back to relations in the universe). A declaration contains a \textbf{variable}, a \textbf{multiplicity}, and an \textbf{expression}.

        \begin{center}
            \begin{tabular}{c c c c}
                $x$ & : & some & $E$ \\
                variable & & multiplicity & expression
            \end{tabular}
        \end{center}
    \end{definition}

    An weak quantifier extends the current scope with new bindings as defined in its \textbf{declarations}. For each of these \textbf{declarations} $d := v~:~m~R$, skolemization does the following:

    Assume $S_\forall$ is a list of all universally quantified variables that are currently in scope.

    \begin{itemize}
        \item Define the \textbf{skolem relation}, $\textsf{R}_v$ of arity $\text{size}(S_\forall) + \text{arity}(v)$ (every element of $S_\forall$ should be unary).
        \item Define our \textbf{skolem expression} $\textsf{E}_v$ which will replace every occurrence of $v$ further down the AST.
        \item Approximate an \textbf{upper bound} for the \textbf{skolem relation}.
        \item Calculate \textbf{domain constraints} that ensure the ``inputs'' to $\textsf{R}_v$ are constrained in the same way that their corresponding universally quantified variables are---store these, these will be conjuncted with the formula on the way up the AST once there are no more existential quantifiers. Specifically, these constraints encode multiplicity information that relational bounds cannot infer.
        \item Add \textbf{range constraints} to the formula to ensure that the ``outputs'' of $S$ are constrained in the same way as $v$. Multiplicity information?
    \end{itemize}

    \subsubsection*{Skolem Expression}

    The \textbf{skolem expression}, $\textsf{E}_v$, is what actually replaces $v$. It is the result of joining each non-skolem $a_i \in S_\forall$ with the $\textsf{R}_v$, or $\textsf{E}_v := f~S_\forall~\textsf{R}_v$ where $f$ is defined as follows:
    
    \begin{align*}
        f~[]~X& = X\\
        f~a_i::L~X& = f~L~a_i.X
    \end{align*}

    \subsubsection*{Upper Bound}
    At the moment, Colocolo assumes all quantified variables will be quantified over sets. This way, the upper bound of the \textbf{skolem relation} is simply the cross product of the upper bound of each element in the relation.

    \subsubsection*{Domain Constraints}

    Suppose $R$ is of arity $n$. We call $\textsf{U}$ the universe. Let $I_v = \textsf{R}_v.\textsf{U} \dots \textsf{U}$ where $\textsf{U}$ is joined to $\textsf{R}_v$ $n$ times. Let the $i$th declaration $a_i~:~m_i~R_i$ be the declaration that binds the $i$th variable, $a_i \in S_\forall$. Let $k = \text{length}(S_\forall)$. Our domain constraint is:
    $$I_v ~ \texttt{in} ~ \{a_0~:~m_0~R_0, \dots, a_k~:~m_k~R_k~\vert~\top\}.$$

    Note the expression $\{a_0~:~m_0~R_0, \dots, a_k~:~m_k~R_k~\vert~\top\}$ defines a $k$-ary relation where the $i$th element obeys the $i$th \textbf{declaration} with no other constraints (any relation satisfies $\top$).

    The domain constraints for each subformula in the skolemized formula are lastly conjuncted together at the top level to form the final domain constraint for the whole formula.



    \subsubsection*{Range Constraints}
    Our range constraint is simply $\textsf{E}_v ~\texttt{in}~ m~R$. Let $f' := \textsf{skolemize}(f[\textsf{E}_v / v])$ in $Qv~:~m~R \mid f$ where $Q \in \{\exists, \neg \forall\}$. When our quantifier is $\exists$ the skolemized (sub)formula becomes $(\textsf{E}_v~\texttt{in}~ m~ R) \land f'$, otherwise when $Q$ is $\neg \forall$ we skolemize to $(\textsf{E}_v ~\texttt{in}~ m~R) \to f'$.

    \subsection{An example}
    Let $A$ and $B$ be relations of arity $n$ and $m$ respectively, and let $P$ be a predicate on $A\times B$. Suppose we wish to skolemize the formula
    \[\forall x : \texttt{some}\: A. \: \exists y : \texttt{some}\: B \mid P(x,y).\]
    We first traverse through the universal quantification of $x$, adding $x : \texttt{some}\: A$ to $S_\forall$ and continuing. Next we reach the existentially quantified $y$. Following the above procedure, we define $\textsf{R}_y :_{m + 1} [\{\langle \rangle\}, A\times B]$. We can then define the domain constraint
    \[\textsf{D}_y := \textsf{R}_y .\bigdot\limits_{i = 1}^m \textsf{U}~\texttt{in}~\{x : \texttt{some}\: A \mid \top\},\]
    along with the skolem expression $\textsf{E}_y := x.\textsf{R}_y$. Note that $\textsf{R}_y .\bigdot\limits_{i = 1}^m$ means left-associatively joining \textsf{U} to $\textsf{R}_y$ $m$ times. Lastly, our range constraint is $(\textsf{E}_y~\texttt{in}~\texttt{some}~B)$. As the quantifier is just an $\exists$, our skolemized subformula becomes $(\textsf{E}_y ~ \texttt{in}~\texttt{some}~B) \land P(x, \textsf{E}_y)$. Conjuncted with the top level domain constraints, we obtain the total skolemized formula
    \[\textsf{R}_y .\bigdot\limits_{i = 1}^m \textsf{U}~\texttt{in}~\{x : \texttt{some}\: A \mid \top\} \land (\textsf{E}_y ~ \texttt{in}~\texttt{some}~B) \land P(x, \textsf{E}_y).\]
    Expanding this out, we get
    \[\textsf{R}_y .\bigdot\limits_{i = 1}^m \textsf{U}~\texttt{in}~\{x : \texttt{some}\: A \mid \top\} \land (x.\textsf{R}_y ~ \texttt{in}~\texttt{some}~B) \land P(x, x.\textsf{R}_y).\]

    \section{Reduction to Sat}

    \section{Benchmarks}

    \subsection{Meow: Compiling Kodkod to Ocelot}
  \end{document}
 